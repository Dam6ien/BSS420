\documentclass{article}
\usepackage[margin=1.25in]{geometry}
\usepackage{natbib}
\bibliographystyle{achemso}

\begin{document}

\title{\vspace{-5em} BSS 410 Project Proposal\\\textit{`e casa'}\\ Eco-friendly Sustainable Residential House System} % D: We can take the e casa out of the title if. I like the no caps but It doesn't look good 
\author{Group 6}
\date{\today}
\maketitle

\section{Problem Statement}
  Households consume and produce a variety of products and wastes. Products may take the form of food items, municipal water, electricity and other consumables which are used and then discarded, exiting the household as waste. For example, food clippings from preparing dinner or paper-based packaging is often discarded into dustbins along side plastic waste to be collected and transported elsewhere. Water used at bathroom sinks is disposed of and sent into the municipal water system. Each type of waste generated by a typical residential household requires proceeding systems that will either store, repurpose or discard the waste.
  
  A recent trend has made the consumer market ecologically aware of their own waste generation and many systems have been created to reduce the total waste output of a household. Such products include grey water systems, recycling bins and compost containers. 
  
  Of the available products that consumers may purchase, non of them provide a financial benefit to the consumer, that is, each systems reduces waste but does not focus on increasing value. It is the assumption of this project that the inability of each system to generate a financial benefit is that each systems functions unilaterally.
    
\section{Deliverable}
The aim of this project is to create a System of Systems (SoS), that will not only reduce the waste produced by a generic household but provide a source of profit generation for the consumer. The final deliverable is in the form of a report which shall detail the design processes and analysis of the proposed SoS.

\end{document}
