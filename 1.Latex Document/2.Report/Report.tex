\documentclass[a4paper,11pt,fleqn]{report}

\usepackage{acronym}
\usepackage{amsmath,amssymb,amsfonts}
\usepackage{booktabs}
\usepackage[dvipsnames]{xcolor}
\usepackage[margin=30mm]{geometry}
\usepackage{graphicx}
	\graphicspath{
		{Graphics/}
	}
\usepackage{hyperref}
	\hypersetup{
		colorlinks=true,
		linkcolor=blue,
		filecolor=blue,
		urlcolor=blue,
		citecolor=blue
	}
\usepackage[sort&compress]{natbib}
	\bibliographystyle{apalike}
\usepackage[mark]{gitinfo2}
 \renewcommand{\gitMark}{Branch:\,\gitBranch\,@\,\gitAbbrevHash{}; Author:\,\gitAuthorName; Date:\,\gitAuthorIsoDate~\textbullet{}}
\usepackage{url}

\begin{document}
%===============================================================
% Frontmatter and title page to be added manually at the end
%===============================================================
\thispagestyle{empty}
\begin{center}
{\huge \textit{`e-casa'} - Eco-friendly Household Waste System\\Part II}
\vspace{20mm} \\
{\Large Group 6}
\vfill

A project report in partial fulfilment of the requirements for the subject\\
\vspace{10mm}
{\Large \textsc{Systems Engineering (BSS 410)}} \\
\vfill
%
in the \\
\vspace{20mm}
%
{\Large \textsc{Faculty of Engineering, Built Environment, and \\ 
Information Technology}}\\
%
\vspace{10mm}
{\Large\textsc{University of Pretoria}} \\
%
\vfill
%
\today
\end{center}

\pagenumbering{roman}

\tableofcontents
\listoffigures\addcontentsline{toc}{chapter}{List of Figures}
\listoftables\addcontentsline{toc}{chapter}{List of Tables}

\chapter*{Acronyms}
\addcontentsline{toc}{chapter}{Acronyms}
\begin{acronym}[ABCDEF]
\acro{CTD}{Centre for Transport Development}
\acro{UP}{University of Pretoria}
\end{acronym}

\chapter{Introduction}
\pagenumbering{arabic}
\setcounter{page}{1}
\acresetall
Welcome to writing in \LaTeX~ during your stay in \ac{CTD} at \ac{UP}.
At the start of each chapter we reset all the acronyms so that they are typed out in full the first time they are used in a chapter. 

\section{Background}

There is usually some introductory text before you just jump in with \textbackslash\texttt{section\{\ldots\}} and \textbackslash\texttt{subsection\{\ldots\}} commands.

\subsection{Subsection}
You have to ensure that there is proper flow through your document. 
One suggestion is to plan your document by only adding the various section and subsection headings. 
You should have flow through them. 
And once there is logical flow at the structural level, you can start populating the different sections.

In \LaTeX, when you want to start with a new paragraph, you simply leave open a line in the source \texttt{$\star$.tex} file, without adding specific line break commands `\textbackslash\textbackslash'. 
The formatting of the paragraphs so that each subsequent paragraph starts with the first line indented, and \emph{no open space between lines}, is sorted out by \LaTeX~during compilation.

\chapter{Literature review}
\acresetall
So at the start of a new chapter, the first use of the acronym \ac{UP} should be written in full again.
I doubt \citet{ar:Manson2006} will think this is a rigorous review.


\chapter{Model}
\acresetall
And here is some mathematical formula expressed in~\eqref{eq:example}.
\begin{align}
y & = mx + c \label{eq:example}
\end{align}
where $m$ is the gradient of the line.

\chapter{Results and discussion}
\acresetall

\chapter{Conclusion}
\acresetall
This is the chapter where you add \emph{concluding} remarks. It is not a summary.


\bibliography{Example}


\appendix
\chapter{Some data as appendix}

\end{document}
