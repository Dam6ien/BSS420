\documentclass{article}
\usepackage[margin=1.25in]{geometry}
\usepackage{natbib}
\bibliographystyle{achemso}

\begin{document}

\title{\vspace{-5em} BSS 410 Project Proposal\\\textit{`e-casa'} - Eco-friendly Household Waste System} % D: We can take the e casa out of the title if. I like the no caps but It doesn't look good 
\author{Group 6}
\date{\today}
\maketitle

\section*{Problem Statement}
 Households consume and produce a variety of products and wastes. Products may take the form of food items, municipal water, electricity and other consumables which are used and then discarded, exiting the household as waste. For example, food clippings from preparing dinner or paper-based packaging is often discarded into dustbins along side plastic waste to be collected and transported elsewhere. Water used at bathroom sinks is disposed of and sent into the municipal water system. Each type of waste generated by a typical residential household requires proceeding systems that will either store, repurpose or discard the waste.
  
  Recent trends have made the consumer market ecologically aware of their own waste generation and many systems have been created to reduce the total waste output of a household. Such products include grey water systems, recycling bins and compost containers. There is a growing need for these systems but presently no solution on the market that integrates all systems into one comprehensive system for homeowners to make use of.
  
  Because these systems operate independently, the outputs of one system are often not used as inputs to another. Integrating these systems to operate together could provide more value to the user and further reduce household waste. For e.g. using food clippings to create a compost heap and grow plants and foods in one's garden instead of disposing of the food clippings in a general waste bin and purchasing compost from a store.
    
\section*{Deliverable}
The aim of this project is to create a System of Systems (SoS), that will not only reduce the waste produced by a generic household but do so by increasing value to the user. The final deliverable for this project will be a report that details the needs analysis, conceptual design, feasibility and risk assessment for the SoS. The coneptual design of this SoS will be done with the use of \textit{Core 9 Software} to capture all identified needs accurately and produce system design documentation.

\end{document}
